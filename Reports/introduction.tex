The multi-user multiple-input multiple-output (MU-MIMO) based transmission gained popularity with the invention of multiple antenna transmission which provides huge throughput enhancement utilizing the spatial dimension. The user selection for MU-MIMO demands the channel orthogonality so as to decouple the transmitted data for each users utilizing the spatial dimension. Spencer {\it et al.} \cite{spencer2004zero} pioneered the work on selecting the users for MU-MIMO based transmission. Search algorithm based on semi-orthogonal user selection (SUS) using Gram Schmidt QR is proposed in \cite{sus2006zfbf}. Algorithms based on block diagonalization (BD) of the stacked channel vectors by projecting successively on to the null space are discussed in \cite{shen2006low}, \cite{youtuan2007improved} and \cite{dimic2005downlink}. Precoder design in an iterative manner is proposed in \cite{traniterative} for BD based user search. The paper also addressed the complexity issues involved in diagonalizing the large concatenated matrix. The user selection schemes for MU-MIMO based transmission scheme are classified and analysed briefly in \cite{zhang2007user} which aimed at minimizing the average beamforming power. Scheduling based on genetic search which includes different criteria were discussed in \cite{genetic_search}.



The scheduling schemes discussed earlier assume the infinite buffer model which in reality is not usually true. Algorithms with queueing model assumption and buffer limitations are discussed in the literature \cite{queuevsinfo}. It is also shown that the capacity achieving scheduling based on information theory provides the fairness among users thereby provides the stable queue length \cite{queuevsinfo}. The selection based on the volume of the enclosed vectors is discussed briefly in \cite{jin2010novel}, \cite{wang2010maximum} and \cite{ko2012determinant}. The precoder design for MU-MIMO based transmission uses ZF based design as discussed in \cite{wiesel2008zero}.
