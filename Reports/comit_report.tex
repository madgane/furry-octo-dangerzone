
\documentclass[a4paper,11pt,draft,fleqn,onecolumn,draft]{article}


\usepackage[margin=2cm]{geometry}
\usepackage{algorithm2e}

\usepackage{dsfont}
%\usepackage{xfrac}
\usepackage[dvips]{graphicx}

\begin{document}

\input{./../References/shortcut_commands}

\section{Introduction}

\section{System Model}


The system model considers \me{\mscrmxy{N}{B}} BSs, each has \me{\mscrmxy{N}{T}} transmit antennas and \me{K} users with a single receive antenna. The set \me{\mcxy{U}{b}} denotes the set of users linked to BS \me{b} where \me{b \in \{1,2,\dotsc,\mscrmxy{N}{B}\}}. The set \me{\mc{U}} given by \me{\mcxy{U}{b} \sset \mc{U}} represents the total users in the system as represented by \me{\card{\mc{U}} = K}. The set \me{\mcxy{B}{k} \sset \mc{B} = \{1,2,\dotsc,\mscrmxy{N}{B}\}} represents the set of BSs transmitting for the user \me{k}. The cardinality of the set \me{ \card{\mcxy{B}{k}} \geq 1} represents the coordinated transmission from more than one BS in the system. The users are statically assigned to a BS in the single BS transmission scenario based on the pathloss measures. The received signal \me{\msclxy{y}{k}} of the user \me{k} consisting of both inter cell and intra cell interference is given by
\begin{eqnarray}
\msclxy{y}{k} = \underbrace{\sum_{\mtext{b} \inm \mcxy{B}{k}} \mvecxyz{h}{b}{k} \mvecxyz{x}{b}{k}}_{\mathclap{\text{desired term}}} \quad + \quad \overbrace{\sum_{\mtext{b} \inm \mcxy{B}{k}} \mvecxyz{h}{b}{k} \sum_{\mathclap{\mtext{i} \inm \mcxy{U}{b} \bs \mtext{k}}} \mvecxyz{x}{b}{i}}^{\mathclap{\text{intra cell interference terms}}} \quad + \quad \overbrace{\sum_{\mathclap{\mtext{c} \inm \mc{B} \bs \mcxy{B}{k}}} \mvecxyz{h}{c}{k} \sum_{\mtext{j} \inm \mcxy{U}{c}} \mvecxyz{x}{c}{j}}^{\mathclap{\text{inter cell interference terms}}} \quad + \quad \msclxy{n}{k}
\label{sm-e1}
\end{eqnarray}

where the vector \me{\mvecxyz{x}{b}{k} \inm \mathds{C}^{\mscrmxy{N}{T}}} represents the transmitted symbol from the BS \inmt{b} to user \inmt{k}, \me{\msclxy{n}{k} \sim \mc{CN}(0,\mtext{N}_0)}, and \me{\mvecxyz{h}{b}{k} \inm \mathds{C}^{1 \times \mscrmxy{N}{T}}} denotes the channel (including pathloss) between the BS \inmt{b} to the user \inmt{k}.

The transmitted symbol \me{\mvecxyz{x}{b}{k}} for the user \inmt{k} from BS \inmt{b} is given by \me{\mvecxyz{x}{b}{k} = \mvecxyz{m}{b}{k} \, \msclxy{d}{k}} where \me{\mvecxyz{m}{b}{k}} is the precoder used by the BS \inmt{b} for user \inmt{k} and \me{\msclxy{d}{k}} denotes the data meant for user \inmt{k} with \me{\mbf{E} [\, \card{d}^2 ] = 1}. The total power used by the transmitter is given by
\begin{equation}
\sum_{k \inm \mcxy{U}{b}} \mrm{Tr} \left ( \mvecxyz{x}{b}{k} \, \mvecxyz{x}{b}{k}^{\mrm{H}} \right ) \leq \mrm{P}_{t}
\label{sm-e2}
\end{equation}

The precoding scheme is based on weighted minimum mean squared error (W-MMSE) scheme discussed in \cite{wmmse_shi} or by combined zero-forcing (CZF) scheme by stacking the channel of users in the transmission set of all BSs in \me{\mc{B}} as discussed in \cite{spencer2004zero}. Once precoders are defined, power allocation is performed over precoders designed by CZF scheme to either maximize the sum capacity or to minimize the expected queue size. W-MMSE based precoding scheme is also analyzed in this paper to bring out the performance comparison. W-MMSE scheme provides joint design of precoder and power allocation for each users in the network.

Scheduling schemes are compared with the well established schemes discussed in \cite{sus2006zfbf,zhang2007user} which aims at selecting least correlated users for the transmission set for MU-MIMO. The selection process is performed in an iterative manner by selecting the first user based on the channel norm.
\begin{eqnarray}
\mbf{N} &=& \mbf{I} - \mbf{U} ( \, \mbf{U}^\mrm{H} \mbf{U} \, )^{-1} \mbf{U}^\mrm{H} \label{sm-e3} \\
\mbf{U} &=& \matscont{\mvecxyz{h}{b}{x}^\mrm{T} \; \mvecxyz{h}{b}{y}^\mrm{T}} \fall x,y \inm \mc{S}_b \text{ where } b \inm \mc{B} \nonumber
\end{eqnarray}
The successive users are chosen by selecting the user with the highest projection gain over the null space formed by the channel vectors of the already chosen users at BS \me{b} as given by
\begin{equation}
j = \argmax_i \gnorm{\mbf{N}^\mrm{H} \mvecxyz{h}{b}{i}^\mrm{T}} \, \fall i \inm \mc{U}_b.
\end{equation}
The user \me{j} is then selected as the next user in the transmission set and the process is repeated until the transmission user set \me{\card{\mc{S}_b} = N_\mrm{T}}. The selection schemes discussed in \cite{jin2010novel,ko2012determinant} performs similar to the one discussed earlier but with different interpretation which is based on maximizing the volume formed by the user channel vectors. The overall complex multiplications involved is given as
\begin{equation}
\approx \left ( \, N^3_\mrm{T} (N_\mrm{T} - 1) + N^2_\mrm{T} \, \right ) \; \card{\mc{U}_b} + \sum_{i = 1}^{N_\mrm{T} - 1} 2i N^2_\mrm{T} + i^2( \, i + N_\mrm{T} \,)
\label{sm-e4}
\end{equation}
where the terms inside summation are meant for the null space calculations.

The scheduling schemes discussed are not limited to single receive antenna; it can be extended for multi antenna by treating each spatial streams as virtual users using singular value decomposition (SVD) over the channel matrix of users in \me{\mc{U}}. The precoder design using W-MMSE scheme is straight forward with the transmission user set selected based on scheduling schemes as it optimizes both transmit and receive beamformers jointly. The zero-forcing precoding is performed in an iterative manner by fixing the receive beamformers using MMSE receivers as discussed in \cite{antti_user_selection}.


\section{Single BS User Scheduling}

The user selection scheme for MU-MIMO transmission performs better when the user \inmt{k \in \mcxy{U}{b}} channels are uncorrelated. The uncorrelated channel constraint helps in decoupling the users data streams with the help of precoders thereby providing interference free transmission. The selection of users with two different objective is studied in this section namely, capacity achieving and fairness based queue size reduction. The channel represented by \me{\mvecxyz{h}{b}{k}} is given by \me{\mvecxy{h}{k}} by dropping the subscript corresponds to BS \me{b = 1}.

\subsection{Max-Throughput based User Scheduling} \label{mtbus}

The selection methods with the objective of maximizing the overall throughput is considered in this section. The algorithms mentioned here are classified based on the performance and complexity. The complexity involved is lowered by reducing the operations involved in calculating the metric used for comparison.

\subsubsection{Eigen vector based User Selection}


This section addresses the question of how the performance can be improved by knowing the channel inner product. The inner product matrix is already discussed in the optimization objective in (\ref{opt_eqn}). Matrix $\mathbf{U}^H \mathbf{U}$ provides the inner product between all the channel vectors in $\mathcal{K}$. The objective as posed is to maximize the volume of the parallelopipe by selecting the binary matrix $\mathbf{X}$  with ones at the desired users. The metric used to determine the degenerate subspace is given by
\begin{ceq}
\mathbf{C}_{i,j} = \det{\left ( \left [\mathbf{h}^T_i \ \mathbf{h}^T_j]^H [\mathbf{h}^T_i \ \mathbf{h}^T_j \right ] \right )}
\label{mca_1:1}
\end{ceq}
where $i,j$ correspond to the user indices. The entry $\mathbf{C}_{i,j}$ provides a measure of the volume enclosed by the vectors $\mathbf{h}_i$ and $\mathbf{h}_j$. The matrix $\mathbf{C}$ which is symmetric, can be decomposed into a diagonal form by unitary matrix $\mathbf{E}$ which is then represented by $\mathbf{E} \mathbf{\Lambda} \mathbf{E}^H$. Matrix $\mathbf{E}$ is represented as the matrix of $\mathbf{e}$ vectors and the diagonal matrix $\mathbf{\Lambda}$ contains the corresponding eigen-values in the diagonal.

The selection is carried out by selecting the user whose channel vector forms the maximum volume with all other users in the system and the existing users in $\mathcal{S}$. This is achieved by selecting the maximum index of the maximum eigen vector of the matrix $\mathbf{C}$ as discussed in \cite{saaty2008decision}. The index of the maximum element in $\mathbf{e}_{\lambda_{max}}$ corresponds to the user index to be selected for $\mathcal{S}$ where $\lambda_{max}$ is the index corresponding to the maximum eigen-value in the diagonal elements of $\mathbf{\Lambda}$.

Once the set $\mathcal{S} \neq \{ \emptyset \}$, (\ref{mca_1:1}) is modified to include the channel vectors $\mathbf{h}_k$ of users in the set $\forall \ k \in \mathcal{S}$ as
\begin{eqnarray}
\mathbf{S} &=& [\mathbf{h}^T_a \ \mathbf{h}^T_b \ ... \ \mathbf{h}^T_f] \ \forall \ a,b,...,f \in \mathcal{S} \\
\mathbf{C}_{i,j} &=& \det{\left ( \left [ \mathbf{S} \ \mathbf{h}^T_i \ \mathbf{h}^T_j]^H [\mathbf{S} \ \mathbf{h}^T_i \ \mathbf{h}^T_j \right ] \right ) }
\label{mca_1:2}
\end{eqnarray}
with the updated matrix $\mathbf{C}$, the same procedure is carried out as earlier to arrive at the next user. This process is then repeated by updating (\ref{mca_1:2}) to find the remaining users which is explained briefly in Algorithm \ref{algorithm_3}, where $e^l_{\lambda_{max}}$ is the $l^{\mathrm{th}}$ entry in the vector $\mathbf{e}_{\lambda_{max}}$. Since initial selection is based on the pairwise determinant metric, users are selected based on maximizing the area over the remaining set of users in $\{ \mathcal{K} \backslash \mathcal{S} \}$ at each iteration of user search.

%\begin{algorithm}[H]
%\STATE{\textbf{Initialize:} $\mathcal{S} = \{\emptyset\}$}
%\STATE{\textbf{Formulate:} $\mathbf{C}$ based on (\ref{mca_1:1})}
%\STATE{\textbf{Search:} $\displaystyle i = \arg \max_l | e^l_{\lambda_{max}} | \ \forall \ e^l_{\lambda_{max}} \in \mathbf{e}_{\lambda_{max}}$}
%\STATE{\textbf{Update:} $\mathcal{S} = \mathcal{S} \cup i$}
%\WHILE{$|\mathcal{S}| \leq N_T$}
%\STATE{\textbf{Formulate:} $\mathbf{C}$ based on (\ref{mca_1:2})}
%\STATE{\textbf{Search:} $\displaystyle i = \arg \max_l | e^l_{\lambda_{max}} | \ \forall \ e^l_{\lambda_{max}} \in \mathbf{e}_{\lambda_{max}}$}
%\STATE{\textbf{Update:} $\mathcal{S} = \mathcal{S} \cup i$}
%\ENDWHILE
%\caption{Eigen vector based User Selection}
%\label{algorithm_3}
%\end{algorithm}

\subsubsection{Selection based on Reduced Null Space Gain}


The objective of capacity achieving user selection and the complexity involved in implementing the same is considered. The proposed method provides an alternative way to achieve the null space calculations involved in the selection procedure. User selection based on QR based decomposition requires null space formulation in order to find the orthogonal subspace for the given vector channel were discussed in \cite{zhang2007user,antti_user_selection,traniterative,sun2009eigenmode}.

The user channels are projected on to the null space of the existing users channel vectors in \me{\mc{S}} where \me{\mc{S}} represents the users selected for the current scheduling instant. The null space is approximated by the product of the vertical projection distance from the given channel vector to the existing users channel vectors in \me{\mc{S}}. The channel vector \me{\mvecxy{h}{i} \fall i \inm \mc{U}} is projected on to the matrix \me{\mbf{U}} formed by stacking the normalized channel vectors of the users in \me{\mc{S}}.
\begin{equation}
\mbf{U} = \left [ \, \frac{\mvecxy{h}{i}^\mrm{T}}{\gnorm{\mvecxy{h}{i}}}, \dotsc, \frac{\mvecxy{h}{j}^\mrm{T}}{\gnorm{\mvecxy{h}{j}}} \, \right ], \; \fall i,j \inm \mc{S}
\label{mca2-e1}
\end{equation}
In order to find the vertical projection distance, channel vector of the users in \me{\mc{U}} is projected on \me{\mbf{U}} is  denoted by \me{\mbf{g}}. The vector \me{\mbf{g}} contains the projection gains of the given channel vector onto the normalized channel of the users in \me{\mc{S}} as
\begin{equation}
\mbf{g} = \mbf{U}^\mrm{H} \mvecxy{h}{i}^\mrm{T}, \fall i \inm \mc{U}
\label{mca2-e2}
\end{equation}
The vertical projection distance is given by subtracting the norm of channel vector from the projection metric \me{\mbf{g}}. The null space projection is then obtained by multiplying the vertical projection distance from all the unit vectors from the given channel vector as
\begin{equation}
\msclxy{m}{i} = \prod^{\card{\mc{S}}}_{l = 1} \left ( \, \gnorm{\mvecxy{h}{i}} - \card{g_l} \, \right ), \fall i \inm \mc{U}
\label{mca2-e3}
\end{equation}
where \me{g_l} represents \me{l^\mrm{th}} element in \me{\mbf{g}} and \me{\msclxy{m}{i}} denotes the null space projection of user \me{i} on to the existing user set \me{\mc{S}}.

The product measures the gain achieved by projecting the vector \me{\mvecxy{h}{i}} over the null space of the vectors formed by the channel vectors of users in the set \me{\mc{S}}. The product yields `\me{0}' when the vector \me{\mvecxy{h}{i}} is in the direction of the unit channel vectors in \me{\mbf{U}}. The product will not provide `\me{0}' when it is collinear with any two vectors in \me{\mbf{U}} as given by null space. Even though this method provides an approximation for null space, performance achieved by this scheme is closer to that of the selection scheme achieved by null space based projections discussed in \cite{sus2006zfbf,antti_user_selection,icsps2010}.

The approximation of \eqref{mca2-e2} and \eqref{mca2-e3} in the current scheme provides the approximation for the null space projection discussed in \eqref{sm-e3} providing noticeable reduction in the complexity involved in calculating the metric for each users. The distance metric uses \me{\left ( \gnorm{\mvecxy{h}{i}} - \card{g_l} \right ) } with \me{g_1} being the projection over the unit direction vector. The metric is different from the ideal vertical projection distance which is measured by the distance which is always higher. Since the metric is based on the absolute distance, it provides conservative estimate following the inequality \me{\gnorm{\mvecxy{h}{i}}\sin(\theta) \geq 2\gnorm{\mvecxy{h}{i}}\sin^2\left (\frac{\theta}{2}\right )} over the interval \me{\theta \in \matscont{0,\frac{\pi}{2}}}. The pseudo code is briefed in the Algorithm. \ref{mca2-a1}.

\begin{algorithm}
 \SetAlgoLined
 \DontPrintSemicolon
 \KwIn{\me{\mvecxy{h}{k} \fall k \inm \mc{U} }}
 \KwData{\me{\mc{S} = \emptyset, \, \mbf{U} = [\,]}}
 \While{\me{\card{S} \leq N_\mrm{T}}}{
 \ForEach{\me{ i \inm \{1,2,\dotsc,K\}}}{
 formulate \me{\mbf{U}} using \eqref{mca2-e1} \;
 calculate \me{\msclxy{m}{i}} using \eqref{mca2-e2}, \eqref{mca2-e3} \;
 }
 select user \me{\displaystyle k = \argmax_i \: \msclxy{m}{i}} \;
 \me{\mc{S} = \{ \, \mc{S} \cup k \, \}} \;
 }
 \caption{Selection based on Reduced Null Space Gain}
 \label{mca2-a1}
\end{algorithm}

The performance of this scheme is equivalent to the QR based scheme with the \me{2 \times 1} system where the null space calculation in \eqref{sm-e3} is identical to \eqref{mca2-e3}. The complex multiplications involved in the search algorithm is given by
\begin{equation}
\approx \card{\mc{U}_b} \left ( N^2_\mrm{T} + \sum^{N_\mrm{T}}_{i = 1} (i - 1) \, N^2_\mrm{T} \right ) + \sum^{N_\mrm{T} - 1}_{i = 1} N^2_\mrm{T}
\label{mca2-e4}
\end{equation}
which involves the norm metric calculation instead of null space calculation as in \eqref{sm-e4}.


\subsection{Queue based User Scheduling} \label{qbus}

In this section, we discuss the selection schemes which considers the queue backlogs of each user and aims at minimizing it for each users in the set \me{\mc{U}}. Even though the users are selected based on the objective of minimizing the queues, power allocation based on water-filling (WF-PA) for zero-forcing (ZF) precoders has the objective of maximizing the sum capacity. The WF-PA scheme is replaced with the queue based power allocation scheme which uses queue weighted sum rate maximization objective (QW-PA).

The W-MMSE based precoding scheme is also analyzed in this section for the precoder design with the expected queue minimizing objective. The following section discusses two selection schemes with the objective of reducing the queues namely weighted user selection and percentile proportional fair scheduling scheme.

\subsubsection{Queue weighted User scheduling}


User selection which has the objective of reducing the expected queue of each user select users with higher queue backlogs. The selection strategy based on queue and the channel condition is discussed extensively in \cite{neely2012stability}. The queue based selection based on Lyapunov drift reduction along with power reduction is also discussed. The performance of those schemes are well suited for single user MIMO (SU-MIMO) where the entire spatial dimension is allotted for a single user. 

In case of MU-MIMO, the users sharing spatial dimension will share the power and moreover interfere with each other unless precoder decouples the transmitted data. In order to decouple the transmitted data, the channel of the users in \me{\mc{S}} should be uncorrelated spatially. Once precoders are designed, power allocation for each users is performed based on queues and the channel condition in order to minimize the expected queue length for each users.

Let \me{\mbf{Q}_i(n)} represents the backlog packets at \me{n^\mrm{th}} instant and \me{\mbf{b}_i(n)} represents the transmission bits at \me{n^\mrm{th}} instant for \me{i^\mrm{th}} user. Following the optimization strategy as given in \cite{neely2012stability} based on Lyapunov stability, the objective is to select the user set \me{\mc{S}} which maximizes the following
\begin{eqnarray}
&\displaystyle \max_{\mc{S} \inm \mc{U}} \, \sum_{i \inm \mc{S}}\mbf{Q}_i(n) \, \mbf{b}_i(n) \\
&\displaystyle \mrm{subject \: to,} \quad \card{\mc{S}} \leq N_\mrm{T}
\label{qba1-e1}
\end{eqnarray}  
The above problem is difficult to solve due to the constraint \eqref{qba1-e1} which casts it as a combinatorial search problem.


\subsubsection{Percentile proportional fair scheduling}

\section{Multi BS User Scheduling}

\subsection{Iterative User Scheduling}

\subsection{Coordinate User Scheduling}

\section{Numerical Results}

\subsection{Single BS-US}

\subsection{Multi BS-US}

\section{Conclusion}

\bibliographystyle{ieeetr}
\bibliography{./../Library/kirja_survey}

\end{document}
