
This section discusses the user selection strategy performed over the set \me{\mc{U}_b \sset \mc{U}} which represents the users linked to BS \me{b}. The selection performed over the associated user set \me{\mc{U}_b, \fall b \inm \mc{B}} provides the transmission set \me{\mc{S}_b} of BS \me{b} which needs to be signalled to all BSs in the set \me{\{ \mc{B} \bs b \}}. The selection is carried out in an iterative manner with the limited signaling between the cooperating BSs to maximize the overall sum throughput.

In multi-BS scenario, the available spatial dimension \me{N_\mrm{T}} is shared between the transmission of information to the desired cell users and minimizing the interference created to the neighboring cells. This is performed by sharing the available spatial freedom \me{N_\mrm{T}} over the cooperating BSs in \me{\mc{B}} by dividing the spatial dimension over \me{\card{\mc{B}}} as given by \me{\left \lfloor \frac{N_\mrm{T}}{N_\mrm{B}} \right \rfloor} with the assumption \me{N_\mrm{T} \geq N_\mrm{B}}.

Let \me{\mc{S}_b} and \me{\mc{S}_{b,\mrm{H}}} be initialized with \me{\emptyset, \; \fall b \inm \mc{B}} where the later represents the user set selected during previous iteration for BS \me{b}. The scheduling is performed by a centralized server which updates the transmission user set or distributively by each BS by exchanging the transmission user set \me{\mc{S}_b} between them. To begin with, let us consider BS \me{\hat{b}} select the users from \me{\mc{U}_{\hat{b}}} while satisfying the cardinality constraint mentioned earlier. The selected user set \me{\mc{S}_{\hat{b}}} is broadcasted to all BS in \me{\mc{B}} to perform their respective selections. The exchanged information \me{\mc{S}_{\hat{b}}} is used by all BS in \me{\mc{B}} to select the users for transmission set \me{\mc{S}_j} of BS \me{j} using projection metric.
\begin{equation}
\begin{array}{lcl}
d_k &=& \det \matbcont{\mbf{T}^\mrm{H} \; \mbf{T}} \\
\mbf{T} &=& \matscont{\mbf{T}_\mrm{X} \; \mbf{T}_\mrm{D} \; \mvecxyz{h}{j}{i}}, \fall i \inm \mc{U}_j
\end{array}
\label{sus-e1}
\end{equation}
where \me{\mbf{T}} is estimated using \me{\mbf{T}_{\mrm{X}}} which corresponds to the stacked channel vectors between neighboring users and the BS \me{j} and \me{\mbf{T}_{\mrm{D}}} represents the stacked channel vectors of users already selected for transmission in the same BS \me{j} as given by
\begin{equation}
\begin{array}{lcl}
\mbf{T}_{\mrm{D}} &=& \matscont{\mvecxyz{h}{j}{k} \dotso}, \fall k \inm \mc{S}_j \\
\mbf{T}_{\mrm{X}} &=& \matscont{\mvecxyz{h}{j}{k} \dotso }, \fall k \inm \displaystyle \bigcup_{\mathclap{\fall b \inm \mc{B} \bs j}} \mc{S}_b.
\end{array}
\label{sus-e2}
\end{equation}

The user set \me{\mc{S}_j} is updated with the user having maximum metric \me{\hat{k} = \argmax_k \, d_k} as \me{\mc{S}_j = \setcont{\mc{S}_j \cup \hat{k}}} till the constraint \me{\card{\mc{S}_j} = \left \lfloor \frac{N_\mrm{T}}{N_\mrm{B}} \right \rfloor} is satisfied. This process is performed over all BSs in a sequential manner to complete one iteration. The second and further iterations are carried out for each BS \me{k} by setting \me{\mc{S}_k = \emptyset} and the new set of transmission user set \me{\mc{S}_k} is updated by using \eqref{sus-e1} until the sets are converged. The convergence is guaranteed since the search is over the closet set of channel vectors. The Algorithm \ref{sus-a1} describes the procedure of user selection for multi-BS in an iterative manner.
\begin{algorithm}
 \SetAlgoLined
 \DontPrintSemicolon
 \KwIn{\me{\mvecxyz{h}{b}{i} \fall i \inm \mc{U}_b, \fall b \inm \mc{B}}}
 \KwData{iteration index \me{i = 0}, \me{\mc{S}_b = \emptyset}, \me{\mc{S}_{b,\mrm{H}} = \emptyset}, \me{\fall b \inm \mc{B}}}
 \While{\me{ \matbcont{\bigcup_{b} \mc{S}_{b,\mrm{H}} \neq \bigcup_b \mc{S}_b} \; \& \; \matbcont{\sum_b \card{\mc{S}_{b,\mrm{H}}} \neq 0}, \; \fall b \inm \mc{B}}}{
 i = i + 1 \;
 \ForEach{\me{b \inm \mc{B}}}{
 \me{\mc{S}_b = \emptyset} \;
 \ForEach{\me{k \inm \mc{U}_b}}{
 evaluate \me{d_k} using \eqref{sus-e1} and \eqref{sus-e2}\;
 }
 find user \me{\hat{k} = \argmax_k \; d_k} \;
 \me{\mc{S}_{\hat{b}} = \setcont{\mc{S}_{\hat{b}} \cup \hat{k}}} \;
 }
 \me{\mc{S}_{b,\mrm{H}} = \mc{S}_b,\; \fall b \inm \mc{B}} \;
 }
 \caption{Static user scheduling}
 \label{sus-a1}
\end{algorithm}

Precoder design is performed jointly over all BSs in the system over the transmission user set \me{\mc{S}_k \fall k \inm \mc{B}} using either WMMSE \cite{wmmse_shi} or ZF of the stacked channel vectors of all users. The gain achieved by this selection scheme is equal to that of overloaded precoder design using WMMSE method with all users in the precoder design procedure.
