
This section discusses the user selection strategy performed over the set \me{\mc{U}_b \sset \mc{U}} which represents the users linked to BS \me{b}. The selection performed over the associated user set \me{\mc{U}_b, \fall b \inm \mc{B}} provides the transmission set \me{\mc{S}_b} of BS \me{b} which needs to be signalled to all BSs in the set \me{\{ \mc{B} \bs b \}}. The selection is carried out in an iterative manner with the limited signaling between the cooperating BSs to maximize the overall sum throughput.

In multi-BS scenario, the available spatial dimension \me{N_\mrm{T}} is shared between the transmission of information to the desired cell users and minimizing the interference created to the neighboring cells. This is performed by sharing the available spatial freedom \me{N_\mrm{T}} over the cooperating BSs in \me{\mc{B}} by dividing the spatial dimension over \me{\card{\mc{B}}} as given by \me{\left \lfloor \frac{N_\mrm{T}}{N_\mrm{B}} \right \rfloor} with the assumption \me{N_\mrm{T} \geq N_\mrm{B}}. In order to discuss the scheme, let \me{\mc{S}^{i}_{\cup} = \{ \mc{S}_1 \cup \mc{S}_2 \cup \dotso \cup \mc{S}_{N_\mrm{B}} \}} represents the transmission user set of the network at \me{i^\mrm{th}} iteration and 
\me{\mc{S}^i_{\cup,b} = \{ \mc{S}^i_{\cup} \bs \mc{S}_b \} } represents the transmission user set excluding BS \me{b}.

To begin with, initialize \me{\mc{S}_j \fall j \inm \mc{B}} and \me{\mc{S}^i_{\cup}} to \me{\emptyset} at \me{i = 0}. Let us consider \me{\mc{B}_k} selects the transmission user set \me{\mc{S}_k} from \me{\mc{U}_k} based on QR based selection discussed in \cite{zhang2007user,sun2009eigenmode} or single BS user selection as discussed in Sect. \ref{sbus}. This information is used while selecting \me{\{\, \mc{S}_j \fall j \inm \mc{B} \bs k \, \}} in order to provide efficient transmission set \me{\mc{S}^i_{\cup}} at \me{i = 1} iteration instant. The sets in \me{\{\, \mc{S}_j \fall j \inm \mc{B} \bs k \, \}} is updated by selecting the users from the corresponding user set \me{\mc{U}_j} by the following metric
\begin{eqnarray}
\begin{array}{lll}
\mbf{T}_{jj} &=& \left [ \; \mvecxyz{h}{j}{x} \; \right ], \, \fall x \inm \mc{S}_{j} \\
\mbf{T}_j &=& \left [ \; \mvecxyz{h}{j}{l} \; \right ], \, \fall l \inm \mc{S}^i_{\cup,j} \\
\mbf{T}_{j,m} &=& \left [ \; \mbf{T}_j \quad \mbf{T}_{jj} \quad \mvecxyz{h}{j}{m} \; \right ], \, m \inm \mc{U}_j \\
d_m &=& \det{ \left ( \; \mbf{T}^\mrm{H}_{j,m} \, \mbf{T}_{j,m} \; \right ) }
\end{array}
\label{sus-iter0}
\end{eqnarray} 
where \me{\mbf{T}_j} matrix has the channel between BS \me{j} to the neighboring users in \me{\mc{S}^1_{
\cup},j}, \me{\mbf{T}_{jj}} is the matrix with channel vectors of all the users already selected in BS \me{j}, and \me{d_m} represent the metric of \me{m^{\mrm{th}}} user in \me{\mc{U}_j}. The user set \me{\mc{S}_j} is updated with the user having maximum metric \me{u = \argmax_m \, d_m} as \me{\mc{S}_j = \{ \mc{S}_j \cup u\}} and \me{\mc{S}^i_{\cup}} is also updated with \me{\{\mc{S}^i_{\cup} \cup \mc{S}_j \}} once \me{\card{\mc{S}_j} = \left \lfloor \frac{N_\mrm{T}}{N_\mrm{B}} \right \rfloor} is achieved. This process is performed over all BSs in a sequential manner to complete one iteration. The second and further iterations are carried out for each BS \me{k} by setting \me{\mc{S}_k = \emptyset} and the new set of transmission user set \me{\mc{S}_k} is updated by using \eqref{sus-iter0}. The iterations are carried out till \me{\mc{S}^{i - 1}_{\cup} = \mc{S}^i_{\cup}} is achieved. The convergence is guaranteed since the search is over the closet set of channel vectors. The Algorithm \ref{sus-a1} describes the procedure of user selection for multi-BS in an iterative manner.
\begin{algorithm}
 \SetAlgoLined
 \DontPrintSemicolon
 \KwIn{\me{\mvecxyz{h}{k}{i} \fall i \inm \mc{U}_k, \fall k \inm \mc{B}}}
 \KwData{iteration index \me{i = 1}, \me{\mc{S}^i_\cup = \emptyset}}
 \While{\me{\mc{S}^{i-1}_\cup \neq \mc{S}^i_\cup}}{
 \ForEach{\me{k \inm \mc{B}}}{
 \me{\mc{S}_k = \emptyset} \;
 \While{\me{\card{\mc{S}_k} \leq \left \lfloor \frac{N_\mrm{T}}{N_\mrm{B}} \right \rfloor }} {
 \ForEach{\me{ i \inm \mc{U}_k}}{
 evaluate \me{\msclxy{d}{i}} using \eqref{sus-iter0} \;
 }
 select user \me{\displaystyle u = \argmax_i \: \msclxy{d}{i}} \;
 \me{\mc{S}_k = \{ \, \mc{S}_k \cup u \, \}} \;
 }
 \me{\mc{S}^i_\cup = \{ \mc{S}^i_\cup \cup \mc{S}_k \} } \;
 }
 \me{\mc{S}^{i + 1}_\cup = \mc{S}^{i}_\cup}, \me{i = i + 1} \;
 }
 \caption{Static user scheduling}
 \label{sus-a1}
\end{algorithm}

Precoder design is performed jointly over all BSs in the system over the transmission user set \me{\mc{S}_k \fall k \inm \mc{B}} using either WMMSE \cite{wmmse_shi} or ZF of the stacked channel vectors of all users. The gain achieved by this selection scheme is equal to that of overloaded precoder design using WMMSE method with all users in the precoder design procedure.