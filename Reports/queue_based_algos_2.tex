
The queue stability can be achieved by scheduling users based on their respective queue backlogs and their instantaneous channel state information. Proportional fair scheduling achieves this objective by weighing the rate fairness metric \me{\displaystyle {\mbf{r}_i}/{\mbf{R}_i}} with the queue backlogs \me{\mbf{Q}_i}, where \me{\mbf{r}_i, \: \mbf{R}_i} represents the instantaneous and average rate of user \me{i}. The variants of proportional fair scheduling for multi carrier transmissions with different objectives were discussed in \cite{adaptation_crosslayer}.

The fairness objective discussed in \cite{adaptation_crosslayer} perform well for single antenna transmission where the users are separated orthogonally by time or frequency. In case of MU-MIMO transmission, the performance are limited due to the co-channel interference when users are multiplexed in the spatial dimension. In order to provide scheduling fairness and interference free transmission, users selected based on fairness metric should also have uncorrelated channel vectors for efficient decoupling of transmission using precoders.

The proposed scheme achieves the above mentioned objective by considering two level user selection strategy. In the first level, the transmission set \me{\mc{S}} is selected based on the sorted proportional fair metric with queue as the priority factor \me{f_i = {\mbf{r}_i \mbf{Q}_i} / {\mbf{R}_i}} where \me{f_i} is the \me{i^\mrm{th}} user fairness metric of the vector \me{\mbf{f}}. Let \me{\mbf{f}_{\pi}} represents the sorted fairness metric in the descending order. The user set \me{\mc{F}_\pi} is populated with \me{x} percentile users based on sorted fairness metric \me{\mbf{f}_\pi} as
\begin{equation}
\mc{F}_\pi = \left \lbrace \: i \mid \arg_i \: \msclxyz{f}{\pi}{i} \geq \msclxyz{f}{\pi}{x}, \fall i \inm \mc{U} \: \right \rbrace
\end{equation}
where \me{x = 50} corresponds to \me{50 \% \: \text{ile}} user selection.

The second level of user selection is performed over the subset \me{\mc{F}_\pi} which has users with better fairness metric over \me{\mc{F}^c_\pi = \mc{U} \bs \mc{F}_\pi}. The user selection is carried out based on queue weighted user scheduling discussed in Sect. \ref{weighted-queue-sched} over the reduced user set \me{\mc{F}_\pi}. The selection procedure achieves fairness objective from the initial user selection and the spatial separation based on the weighted queue user scheduling scheme.
