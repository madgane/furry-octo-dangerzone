
This section discusses the co-ordinate transmission scheme for users in the cell-edge by associating users in \me{\mc{U}} to BSs in \me{\mc{B}} based on instantaneous channel realizations. Users with proportional rate fairness has been dealt in \cite{antti_coord_user_selection} where many variants of user selection for co-ordinate transmission are proposed. The current work addresses the user selection scheme for co-ordinate transmission by selecting users for \me{\mc{S}_b \fall b \inm \mc{B}} in an iterative manner.

Let \me{\mc{S}_b} and \me{\mc{S}_{b,\mrm{H}}} are initialized to \me{\emptyset, \fall b \inm \mc{B}} where \me{\mc{S}_{k,\mrm{H}}} represents the transmission user set from \me{(i-1)^{\mrm{th}}} iteration for BS \me{b}. The user selection is carried out in a centralized server with the transmission set \me{\card{\mc{S}_b} \leq \left \lfloor \frac{N_\mrm{T}}{N_\mrm{B}} \right \rfloor} for each BS \me{b \inm \mc{B}}. The first user is selected by sorting the channel gains of each users in \me{\mc{U}} with each BSs in \me{\mc{B}} and the user \me{\hat{k}} with highest norm is assigned to the corresponding BS \me{\hat{b}}.
\begin{equation}
\mc{S}_{\hat{b}} = \setcont{\mc{S}_{\hat{b}} \cup \hat{k}}
\label{cus-e1}
\end{equation}
where \me{\{\, \hat{b},\hat{k} \,\} \: = \: \argmax_{b,k} \gnorm{\mvecxyz{h}{b}{k}}} represents the selection criterion for the first user. The second and there after will be selected based on finding the user whose channel vector is not on the same subspace (\me{\mvecxyz{h}{\hat{b}}{k}, \fall k}) for users linking to BS \me{\hat{b}} or on to the null space (\me{\mvecxyz{h}{b}{\hat{k}}}) for channel between users with BSs in \me{\{\, \mc{B} \bs \hat{b} \,\}}. The set \me{\mc{S}_j} for BS \me{j} is updated using \eqref{cus-e2} as
\begin{equation}
\begin{array}{lcl}
D_{j,i} &=& \det \left ( \; \mbf{T}^{\mrm{H}} \mbf{T} \; \right ) \\
\mbf{T} &=& \matscont{\mbf{T}_\mrm{X} \; \mbf{T}_\mrm{D} \; \mvecxyz{h}{j}{i}}, \fall i \inm \mc{U}
\end{array}
\label{cus-e2}
\end{equation}
where \me{D_{j,i}} corresponds the the metric evaluated for the user \me{i} when linked to BS \me{j} and \me{\mbf{T}} is evaluated using the following \eqref{cus-e3}
\begin{equation}
\begin{array}{lcl}
\mbf{T}_{\mrm{D}} &=& \matscont{\mvecxyz{h}{j}{k} \dotso}, \fall k \inm \mc{S}_j \\
\mbf{T}_{\mrm{X}} &=& \matscont{\mvecxyz{h}{j}{k} \dotso }, \fall k \inm \displaystyle \bigcup_{\mathclap{\fall b \inm \mc{B} \bs j}} \mc{S}_b
\end{array}
\label{cus-e3}
\end{equation}
where \me{\mbf{T}_\mrm{X}} corresponds to the stacked channel vectors of the channel between the current BS to the neighboring BSs users and \me{\mbf{T}_\mrm{D}} represents the stacked channel vectors of users in the transmission set of the current BS. The selection and association is then carried out by selecting the maximizing metric over the entire \me{\mbf{D}} matrix as 
\begin{equation}
\{ \, \hat{b},\hat{k} \,\} = \argmax_{i,j} \quad D_{i,j} \; \fall i \inm \mc{U} \text{ and } \fall j \inm \mc{B}
\label{cus-e4}
\end{equation}
and \me{\mc{S}_{\hat{b}}} is updated using \eqref{cus-e1}.

Once the users are selected for each BSs with the constraint on \me{\card{\mc{S}_b} \fall b \inm \mc{B}}, the selection is performed again in an iterative manner until the transmission user set converges. This procedure is explained briefly in Algorithm. \ref{cus-a1}
\begin{algorithm}
 \SetAlgoLined
 \DontPrintSemicolon
 \KwIn{\me{\mvecxyz{h}{b}{i} \fall i \inm \mc{U}_b, \fall b \inm \mc{B}}}
 \KwData{iteration index \me{i = 0}, \me{\mc{S}_b = \emptyset}, \me{\mc{S}_{b,\mrm{H}} = \emptyset}, \me{\fall b \inm \mc{B}}}
 \While{\me{ \matbcont{\bigcup_{b} \mc{S}_{b,\mrm{H}} \neq \bigcup_b \mc{S}_b} \; \& \; \matbcont{\sum_b \card{\mc{S}_{b,\mrm{H}}} \neq 0}, \; \fall b \inm \mc{B}}}{
 i = i + 1 \;
 \ForEach{\me{b \inm \mc{B}}}{
 \me{\mc{S}_b = \emptyset} \;
 \ForEach{\me{k \inm \mc{U}}}{
 evaluate \me{D_{b,k}} using \eqref{cus-e2}, \eqref{cus-e3} \;
 }
 \textit{continue} = \textit{true} \;
 \While{continue}{
 find user \me{\hat{k}} and BS \me{\hat{b}} using \eqref{cus-e4} \;
 \eIf{\me{\card{\mc{S}_{\hat{b}}} \leq \left \lfloor \frac{N_\mrm{T}}{N_\mrm{B}} \right \rfloor}}{
 \me{\mc{S}_{\hat{b}} = \setcont{\mc{S}_{\hat{b}} \cup \hat{k}}} \;
 \textit{continue} = \textit{false} \;
 }{
 \me{D_{\hat{b},\hat{k}} = 0} \;
 \textit{continue} = \textit{true} \;
 }
 }
 }
 \me{\mc{S}_{b,\mrm{H}} = \mc{S}_b,\; \fall b \inm \mc{B}} \;
 }
 \caption{Co-ordinate user scheduling}
 \label{cus-a1}
\end{algorithm}

This scheme provides significant improvement over the earlier scheduling scheme discussed in section \ref{sus} based on static user assignment. The improvement is significant only when the pathloss between users \me{\mc{U}} and \me{\mc{B}} are comparable and the users are few in number. As the user count increases over each BS, the coordinate gain vanishes since the gain from multi-user diversity performs significantly better.