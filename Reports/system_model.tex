
The system model considers \me{\mscrmxy{N}{B}} BSs, each has \me{\mscrmxy{N}{T}} transmit antennas and \me{K} users with a single receive antenna. The set \me{\mcxy{U}{b}} denotes the set of users linked to BS \me{b} where \me{b \in \{1,2,\dotsc,\mscrmxy{N}{B}\}}. The set \me{\mc{U}} given by \me{\mcxy{U}{b} \sset \mc{U}} represents the total users in the system as represented by \me{\card{\mc{U}} = K}. The set \me{\mcxy{B}{k} \sset \mc{B} = \{1,2,\dotsc,\mscrmxy{N}{B}\}} represents the set of BSs transmitting for the user \me{k}. The cardinality of the set \me{ \card{\mcxy{B}{k}} \geq 1} represents the coordinated transmission from more than one BS in the system. The users are statically assigned to a BS in the single BS transmission scenario based on the pathloss measures. The received signal \me{\msclxy{y}{k}} of the user \me{k} consisting of both inter cell and intra cell interference is given by
\begin{eqnarray}
\msclxy{y}{k} = \underbrace{\sum_{\mtext{b} \inm \mcxy{B}{k}} \mvecxyz{h}{b}{k} \mvecxyz{x}{b}{k}}_{\mathclap{\text{desired term}}} \quad + \quad \overbrace{\sum_{\mtext{b} \inm \mcxy{B}{k}} \mvecxyz{h}{b}{k} \sum_{\mathclap{\mtext{i} \inm \mcxy{U}{b} \bs \mtext{k}}} \mvecxyz{x}{b}{i}}^{\mathclap{\text{intra cell interference terms}}} \quad + \quad \overbrace{\sum_{\mathclap{\mtext{c} \inm \mc{B} \bs \mcxy{B}{k}}} \mvecxyz{h}{c}{k} \sum_{\mtext{j} \inm \mcxy{U}{c}} \mvecxyz{x}{c}{j}}^{\mathclap{\text{inter cell interference terms}}} \quad + \quad \msclxy{n}{k}
\label{sm-e1}
\end{eqnarray}

where the vector \me{\mvecxyz{x}{b}{k} \inm \mathds{C}^{\mscrmxy{N}{T}}} represents the transmitted symbol from the BS \inmt{b} to user \inmt{k}, \me{\msclxy{n}{k} \sim \mc{CN}(0,\mtext{N}_0)}, and \me{\mvecxyz{h}{b}{k} \inm \mathds{C}^{1 \times \mscrmxy{N}{T}}} denotes the channel (including pathloss) between the BS \inmt{b} to the user \inmt{k}.

The transmitted symbol \me{\mvecxyz{x}{b}{k}} for the user \inmt{k} from BS \inmt{b} is given by \me{\mvecxyz{x}{b}{k} = \mvecxyz{m}{b}{k} \, \msclxy{d}{k}} where \me{\mvecxyz{m}{b}{k}} is the precoder used by the BS \inmt{b} for user \inmt{k} and \me{\msclxy{d}{k}} denotes the data meant for user \inmt{k} with \me{\mbf{E} [\, \card{d}^2 ] = 1}. The total power used by the transmitter is given by
\begin{equation}
\sum_{k \inm \mcxy{U}{b}} \mrm{Tr} \left ( \mvecxyz{x}{b}{k} \, \mvecxyz{x}{b}{k}^{\mrm{H}} \right ) \leq \mrm{P}_{t}
\label{sm-e2}
\end{equation}

The precoding scheme is based on weighted minimum mean squared error (W-MMSE) scheme discussed in \cite{wmmse_shi} or by combined zero-forcing (CZF) scheme by stacking the channel of users in the transmission set of all BSs in \me{\mc{B}} as discussed in \cite{spencer2004zero}. Once precoders are defined, power allocation is performed over precoders designed by CZF scheme to maximize the sum capacity. The power allocation is performed jointly while designing precoders in W-MMSE scheme.

The scheduling schemes discussed are not limited to single receive antenna; it can be extended for multi antenna by treating each spatial streams as virtual users using singular value decomposition (SVD) over the channel matrix of users in \me{\mc{U}}. The precoder design using W-MMSE scheme is straight forward with the transmission user set selected based on scheduling schemes as it optimizes both transmit and receive beamformers jointly. The zero-forcing precoding is performed in an iterative manner by fixing the receive beamformers using MMSE receivers as discussed in \cite{antti_user_selection}. 